\documentclass[11pt]{article}

\setlength{\oddsidemargin}{0in}
\setlength{\textwidth}{6.5in}
\setlength{\topmargin}{-0.5in}
\setlength{\textheight}{8.75in}
\setlength{\parindent}{0pt}
\setlength{\parskip}{6pt}

\usepackage{fancyhdr}
\pagestyle{fancy}
\lhead{HW2}
\rhead{Prudhvi Reddy Araga}

\usepackage{epsfig,graphicx}

\usepackage{amsmath}

\usepackage{clrscode3e}

\begin{document}

\thispagestyle{plain}

\begin{center}
{\Large \bf CIS 606 \hfil Homework 2 \hfil Fall 2020} \\
\end{center}

\vskip 1in 

\centerline{\includegraphics[width=3in]{Photo.jpg}}

\vskip 0.5in 

\begin{center}
\begin{tabular}{ll}
{\bf Name:}     & {\bf Prudhvi Reddy Araga } \\ \\
{\bf Login ID:} & {\bf praraga }   
\end{tabular}
\end{center}

\newpage

\begin{enumerate}

\itemsep 0.35in
 
\item

$F(n) = \Omega (g(n))$ implies $g(n) = $0$ (f(n))$ \\

Answer: True \\

By definition\\
$f(n) = \Omega (g(n))$ which implies  $0 <= c.g(n) <= f(n)$ \\
$g(n) = $0$ (f(n))$ which implies $0 <=  g(n) <= c.f(n)$ \\\\
Let us assume that $f(n) = 100n^2$ , $g(n) = n^2$ \\\\
$f(n) >=  c.g(n)$ \\
$100.n^2 >=  c.n^2$\\
Consider the constant c = 50 \\\\
$100n^2 >= 50n^2$ \\
$2 >= 1$ \\\\
$g(n) = $0$ (f(n))$ which is equal to \\\\ 
$c.f(n) >= g(n)$ \\ 
$c.100.n^2 >=  n^2$\\
$50.100.n^2 >= n^2$ \\
$5000 >= 1$ \\

Based on the above equations $F(n) = \Omega (g(n))$ implies $g(n) = $0$ (f(n))$ is true.
   
\end{enumerate}

\end{document}

