\documentclass[11pt]{article}

\setlength{\oddsidemargin}{0in}
\setlength{\textwidth}{6.5in}
\setlength{\topmargin}{-0.5in}
\setlength{\textheight}{8.75in}
\setlength{\parindent}{0pt}
\setlength{\parskip}{6pt}

\usepackage{fancyhdr}
\pagestyle{fancy}
\lhead{HW1}
\rhead{Prudhvi Reddy Araga}

\usepackage{epsfig,graphicx}

\usepackage{amsmath}

\usepackage{clrscode3e}

\begin{document}

\thispagestyle{plain}

\begin{center}
{\Large \bf CIS 606 \hfil Homework 1 \hfil Fall 2020} \\
\end{center}

\vskip 1in 

\centerline{\includegraphics[width=3in]{Photo.jpg}}

\vskip 0.5in 

\begin{center}
\begin{tabular}{ll}
{\bf Name:}     & {\bf Prudhvi Reddy Araga } \\ \\
{\bf Login ID:} & {\bf praraga }   
\end{tabular}
\end{center}

\newpage

\begin{enumerate}

\itemsep 0.35in
 
\item

$F(n) = \Omega (g(n))$ implies $g(n) = $0$ (f(n))$

Answer: True

By definition

0 \textless{}= c. g(n) \textless{}= f(n)


Hence the above notation is true.

      Using the master theorem in Chapter 4, we can 
      get $T(n) = \Theta (\log{} n)$.

   
\end{enumerate}

\end{document}

